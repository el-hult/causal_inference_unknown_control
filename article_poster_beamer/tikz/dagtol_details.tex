\newcommand{\xmin}{1e-15}
\newcommand{\xmax}{1e2}

\begin{subfigure}[b]{0.4\textwidth}
%\centering
\begin{tikzpicture}[baseline]
    \pgfplotstableread[col sep=comma]{./data/dagtol_pgfplots.csv}{\datatable};
    \begin{loglogaxis}[
        width=\textwidth,
        height=0.7\textwidth,
        xlabel={$\dagTolerance$},
        ylabel={$|\hat{\averageCausalEffectTarget}(\dagTolerance)|$},
        xmin=\xmin,
        xmax=\xmax
    ]
    \foreach \k in {0,1,...,19}
        \addplot [mark=none,color=black] table [x=dag_tolerance, y=ace_abs-\k] {\datatable};
    \end{loglogaxis}
\end{tikzpicture}
\subcaption{The average causal effect estimated for various $\dagTolerance$. Absolute value imposed to allow log-log-plot.}
\end{subfigure}
\hspace{0.1\textwidth}
%
%
%
%
\begin{subfigure}[b]{0.4\textwidth}
\begin{tikzpicture}[baseline]
    \pgfplotstableread[col sep=comma]{./data/dagtol_pgfplots.csv}{\datatable};
    \begin{loglogaxis}[
        width=\textwidth,
        height=0.7\textwidth,
        xlabel={$\dagTolerance$},
        ylabel={$|\averageCausalEffect - \averageCausalEffectNumeric(\dagTolerance)|$},
        xmin=\xmin,
        xmax=\xmax
    ]
    \foreach \k in {0,1,...,19}
        \addplot [mark=none,color=black] table [x=dag_tolerance, y=ace_abs_err-\k] {\datatable};
    \end{loglogaxis}
\end{tikzpicture}
\subcaption{The absolute error in the estimate of the causal effect. Smilar to figure \ref{fig:epsilon-limit}.}
\end{subfigure}


\vspace{.1\textwidth}
%
%
%
\begin{subfigure}[b]{0.4\textwidth}
\begin{tikzpicture}[baseline]
    \pgfplotstableread[col sep=comma]{./data/dagtol_pgfplots.csv}{\datatable};
    \begin{loglogaxis}[
        width=\textwidth,
        height=0.7\textwidth,
        xlabel={$\dagTolerance$},
        ylabel={$\hFun(\hat{\semCoeffMat}(\dagTolerance))$},
        xmin=\xmin,
        xmax=\xmax
    ]
    \foreach \k in {0,1,...,19}
        \addplot [mark=none,color=black] table [x=dag_tolerance, y=h_notears-\k] {\datatable};
    \end{loglogaxis}
\end{tikzpicture}
\subcaption{The constraint function $\hFun$ at the numerical approximation of the $\dagTolerance$-almost \DAG{} $\semCoeffOpt$. If the numerical solver is good and $\dagTolerance \leq \dagToleranceMax$, we should have $\hFun(\hat{\semCoeffMat(\dagTolerance}) \approx \dagTolerance$, which is what we observe down to circa $10^{-12}=\augLagConstraintTol$, the tolerated constraint violation of Algorithm~\ref{algo:augLag}. We can also see that when $\dagTolerance > \dagToleranceMax$, the solution does not depend on $\dagTolerance$.}
\label{fig:sensitivity_details:hfun}
\end{subfigure} 
\hspace{0.1\textwidth}
%
%
%
%
\begin{subfigure}[b]{0.4\textwidth}
\begin{tikzpicture}[baseline]
    \pgfplotstableread[col sep=comma]{./data/dagtol_pgfplots.csv}{\datatable};
    \begin{loglogaxis}[
        width=\textwidth,
        height=0.7\textwidth,
        xlabel={$\dagTolerance$},
        ylabel={$\norm{\semCoeffMat -  \hat{\semCoeffMat}(\dagTolerance) }_{\infty}$},
        xmin=\xmin,
        xmax=\xmax
    ]
    \foreach \k in {0,1,...,19}
        \addplot [mark=none,color=black] table [x=dag_tolerance, y=max-metric-\k] {\datatable};
    \end{loglogaxis}
\end{tikzpicture}
\subcaption{The maximum error in the point estimate of the adjacency matrix $\semCoeffMat$. The results indicate $\dagTolerance \to 0$ is a necessary condition to retrieve the true \DAG{}-matrix $\semCoeffMat$, but numerical precision limits this convergence.}
\end{subfigure}
