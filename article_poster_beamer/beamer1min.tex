\documentclass[aspectratio=1610,10pt]{beamer}
%%%%%%%%%%%%%%%%%%%%%%%%%%%%%%%%%%%%%%% Theming
\usetheme[progressbar=frametitle,numbering=fraction]{metropolis}
\addtobeamertemplate{frametitle}{}{%
\begin{textblock*}{100mm}(.85\textwidth,-1cm)
\includegraphics[width=.65cm,trim={0 1.5cm 1.5cm 0}]{figures/UU_logo_vit.pdf}
\end{textblock*}}

\definecolor{black_}{RGB}{25, 25, 25}
\definecolor{blue_}{RGB}{46,131,191}
\definecolor{green_}{RGB}{46,191,106}
\definecolor{uured}{RGB}{191,45,56}
\definecolor{uudarkgrey}{RGB}{130,130,130}
\definecolor{uumidgrey}{RGB}{190,190,190}
\definecolor{uulightgrey}{RGB}{230,230,230}
\setbeamercolor{alerted text}{fg=uured}
\setbeamercolor{example text}{fg=green_}
%\metroset{block=fill}
\usepackage{pgfplots}
\pgfplotscreateplotcyclelist{myColorList}{%
        color=blue_,every mark/.append style={fill=blue_},mark=*\\%
        color=uured,every mark/.append style={fill=uured},mark=square*\\%
        color=green_,every mark/.append style={fill=green_},mark=otimes*\\%
        color=black_,every mark/.append style={fill=black_},mark=diamond*\\%
    }
\pgfplotsset{every axis/.append style={cycle list name=myColorList}}

%%%%%%%%%%%%%%%%%%%%%%%%%%%%%%%%%%%%%%%%



\usepackage{subcaption}
\usepackage{tikz}
\usetikzlibrary{arrows}
\usetikzlibrary{calc}
\usepackage{pgfplots}	
\pgfplotsset{compat=1.17}
\usepackage{textpos}
\usepackage{amssymb}
\usepackage{amsthm}
\usepackage{mathtools}
\usepackage{booktabs}


\let\definition\relax

% Qutie general things (not math)
\newcommand\red[1]{\textcolor{red}{#1}}

% Qutie general things (math)
\newcommand{\R}{\mathbb R}

%linalg
\newcommand{\kroneckerDelta}{\delta}
\DeclarePairedDelimiter\norm{\lVert}{\rVert}
\DeclareMathOperator*{\vecop}{vec}
\DeclareMathOperator*{\diag}{diag}
\DeclareMathOperator*{\matop}{mat}
\newcommand{\eye}{I}
\DeclareMathOperator*{\tr}{tr}
\newcommand{\hadamard}{\circ}
\newcommand{\kronecker}{\otimes}
\newcommand{\T}{\top}
\newcommand{\vecOne}{\mathbf{1}}

% stats and probability
\DeclareMathOperator*{\cov}{Cov}
\DeclareMathOperator*{\argmin}{arg\,min}
\newcommand{\convp}{\overset{p}{\to}}
\newcommand{\convd}{\overset{d}{\to}}
\newcommand{\E}{\mathbb{E}}
\newcommand{\En}{\mathbb{E}_n}
\newcommand{\Eint}{\widetilde{\mathbb{E}}}
\newcommand{\covint}{\widetilde{\text{Cov}}}
\newcommand{\varint}{\widetilde{\text{Var}}}
\newcommand{\var}{\text{Var}}
\newcommand{\Prob}{\mathbb{P}}
\newcommand{\normal}{\mathcal N}
\newcommand{\permMat}{\mathcal P}
\newcommand{\unitBasisMatrix}{E}


%% Document variables with semantic meaning in this article
\newcommand{\semCoeffMat}{W}
\newcommand{\semCoeffColumn}{w}
\newcommand{\semCoeffMatSet}{\mathcal W}
\newcommand{\semCoeffOpt}{\semCoeffMat_\circ}

\newcommand{\semCoeffEstN}{\semCoeffMat_\nData}
\newcommand{\dagPermutationMatrix}{P}
\newcommand{\semScaleMatrix}{M}
\newcommand{\semVector}{v}
\newcommand{\semVectorOrdered}{\omega}
\newcommand{\semPermutationMatrix}{P}
\newcommand{\semApproximatorFunction}{f}
\newcommand{\semNoise}{e}
\newcommand{\semNoiseCovariance}{\Sigma}
\newcommand{\semInterventionNoise}{\widetilde \semNoise}
\newcommand{\semInterventionNoiseCovariance}{\widetilde \semNoiseCovariance}
\newcommand{\mutilatingMatrix}{Z}
\newcommand{\cofactorMatrix}{C}
\newcommand{\dNodes}{d}
\newcommand{\linearPredictor}{\mathcal L}
\newcommand{\minorMatrix}{m}
\newcommand{\averageCausalEffect}{\gamma}
\newcommand{\averageCausalEffectTarget}{\averageCausalEffect_\circ}
\newcommand{\averageCausalEffectSet}{\Gamma}
\newcommand{\averageCausalEffectEstN}{\averageCausalEffect_\nData}
\newcommand{\averageCausalEffectNumeric}{\hat{\averageCausalEffectTarget}}
\newcommand{\observationalDistribution}{p}
\newcommand{\interventionalDistribution}{\tilde p}
\newcommand{\lossFunc}{\ell}
\newcommand{\outcomeVar}{y}
\newcommand{\decisionVar}{x}
\newcommand{\adjustmentVar}{z}
\newcommand{\adjusted}[1]{\bar{#1}}
\newcommand{\validAdjustmentVar}{\adjusted{z}}
\newcommand{\decisionOptimal}{\hat \decisionVar }
\newcommand{\decisionTrueOptimal}{\decisionVar ^\star}
\newcommand{\decisionSpace}{ \mathcal X}
\newcommand{\dataSet}{\mathcal D}
\newcommand{\nData}{n}
\newcommand{\dagTolerance}{\epsilon}
\newcommand{\dagToleranceMax}{\dagTolerance_\star}
\newcommand{\mEstParameter}{\theta}
\newcommand{\mEstParameterEstN}{\mEstParameter_\nData}
\newcommand{\mEstParameterTrue}{\mEstParameter_\circ}
\newcommand{\mEstParameterSet}{\Theta}
\newcommand{\mEstParametrization}{L}
\newcommand{\mEstLoss}{\ell}
\newcommand{\mEstConstrint}{g}
\newcommand{\mEstCovarianceN}{\mathcal J_\nData}
\newcommand{\regCoefficient}{\beta}
\newcommand{\regCoefficientSet}{B}
\newcommand{\eps}{\varepsilon}
\newcommand{\confidenceLevel}{\alpha}
\newcommand{\hFun}{h}
\newcommand{\qMatrix}{\mathsf{Q}}

% helpers for m-estimation proof
\newcommand{\Un}{U_n}
\newcommand{\Qn}{Q_n}
\newcommand{\Qtrue}{Q_\circ}
\newcommand{\Utrue}{U_\circ}
\newcommand{\Jn}{J_n}
\newcommand{\Kn}{K_n}
\newcommand{\Ktrue}{K_\circ}
\newcommand{\Jtrue}{J_\circ}
\newcommand{\nablatheta}{\nabla}
\newcommand{\PiTrue}{\Pi_\circ}
\newcommand{\PiN}{\Pi_n}

% helpers for augmented lagrangian method
\newcommand{\augLagSlack}{s}
\newcommand{\augLag}{\mathcal L}
\newcommand{\augLagLagMul}{\alpha}
\newcommand{\augLagPen}{\rho}
\newcommand{\augLagPenMul}{\mu}
\newcommand{\augLagPenMax}{\augLagPen_{max}}
\newcommand{\augLagIter}{k}
\newcommand{\augLagMinImprovement}{g}
\newcommand{\augLagContraint}{c}
\newcommand{\augLagConstraintTol}{\eta}

% helpers for discussin misspacified error covariance
\newcommand{\assStruct}{\widehat{\semNoiseCovariance}} % assumed latent covariance structure
\newcommand{\semNoiseScale}{s}
\newcommand{\misspecCond}{\kappa\left(\assStruct^{-1}\semNoiseCovariance\right)}

%repeated acroynoms
\newcommand{\DAG}{\textsc{dag}}
\newcommand{\scm}{\textsc{scm}}
\newcommand{\OLS}{\textsc{ols}}


\begin{document}


\begin{frame}{Inference of Causal Effect when Control Variables are Unknown\footnote{Work by Ludvig Hult and Dave Zachariah, Uppsala University, Contact: \texttt{ludvig.hult@it.uu.se}}}

  \centering
    \begin{tikzpicture}[
      ->,>=stealth',
      shorten >=1pt,
      auto,
      semithick,
      every node/.append style={color=black_},
      every path/.append style={color=uumidgrey},
      ]
      \node[circle,draw, minimum size=20pt,font=\tiny] (Z2) at (2,2) { BMI};
      \node[circle,draw, minimum size=20pt,color=green_] (X) at (-2,0) {\includegraphics[width=1cm]{figures/vitamin-icon-21420.png}};
      \node[circle,draw, minimum size=20pt,color=uured] (Y) at (2,0) {\includegraphics[width=1cm]{figures/toppng.com-coronavirus-covid-19-icon-512x512.png}};
      \node[circle,draw, minimum size=20pt] (Z1) at (-2,-2) {...};
      \node[circle,draw, minimum size=20pt,font=\tiny,align=left] (BP) at (4,2) { Blood \\  Pressure};
      \node[circle,draw, minimum size=20pt,font=\tiny,align=left] (GENDER) at (-3,3) { Gender };
    
      \path (Z1) edge node[above] {?} (X)
      (Z1) edge node[above] {?}  (Y)
      (X) edge node[above] {?}  (Z2)
      (Z2) edge node[above] {?}  (Z1)
      (X) edge[color=blue_] (Y)
      (Y) edge node[above] {?}  (Z2)
      (Y) edge node[above] {?}  (BP)
      (Z2) edge node[above] {?}  (BP)
      (GENDER) edge node[above] {?}  (X)
      (GENDER) edge node[above] {?}  (Y)
      (GENDER) edge node[above] {?}  (BP);
    \end{tikzpicture} 
  \\
    Example: \textbf{Causal effect estimation} in observational database, with \textbf{unknown causal graph}
\end{frame}

%%%%%%%%%%%%%%%%%%%%%%%%%%%%%%%%%%%%%%%%
\begin{frame}{Methods and results}
\begin{columns}
\begin{column}{0.4\textwidth}
  \textbf{\small Model}
  \begin{align*}
    \semVector^{\T} &= (\decisionVar,\outcomeVar,\adjustmentVar_1,...,\adjustmentVar_{d-2})\\
    \semVector &= \semCoeffMat^\T\semVector + \semNoise \\
    \E[\semNoise]&=0 \; \var[\semNoise]=\semNoiseCovariance
  \end{align*}
  \textbf{\small Causal Effect Parameter}
  \begin{align*}
    \averageCausalEffect =   \argmin_{\bar{\averageCausalEffect}} \; \Eint\left[ \big( \Eint[\outcomeVar|\decisionVar]  - \bar{\averageCausalEffect}\decisionVar \big)^2 \right]
  \end{align*}
  \textbf{\small Confidence Interval}
    \begin{tikzpicture}[baseline]
      \pgfplotstableread[col sep=comma]{./data/4node_collider_summary.csv}{\datatable};
      \begin{semilogxaxis}[
        width=\columnwidth,
        height=\columnwidth/2,
        xmin=90,
        xmax=11000,
        legend style={font=\scriptsize, at={(0.5,1.10)},
        anchor=south,legend columns=-1},
        tick label style={font=\tiny}
      ]
        \addplot+ [ only marks, mark=*, mark size=1pt,
            error bars/.cd,
            y dir=both,
            y explicit] table [x=m_obs, y=ace_value, y error=q_ace_standard_error] {\datatable};
        \addplot+ [only marks, mark=*, mark size= 1 pt,
            error bars/.cd,
            y dir=both,
            y explicit] table [x=m_obs, y=ols_value, y error=q_ols_standard_error] {\datatable};
        \addplot [no markers] table [x=m_obs, y=ace_circ] {\datatable};
        \legend{{$\averageCausalEffectSet_{\confidenceLevel,\nData}$},{$\regCoefficientSet_{\confidenceLevel,\nData}$},{$\averageCausalEffectTarget$}}
      \end{semilogxaxis}
    \end{tikzpicture}
  \end{column}
\begin{column}{0.6\textwidth}
  \textbf{Our Contribution}
  \begin{itemize}
    \item Estimating \textcolor{green_}{causal effect parameter} when \textcolor{green_}{unknown control variables}
    \item \textcolor{green_}{Valid confidence interval} for causal effect in linear additive causal model
    \item \textcolor{green_}{Numerical studies} comparing with naive method, and \textcolor{green_}{bootstrap} alternative
  \end{itemize}
\end{column}
\end{columns}
\end{frame}
\end{document}