\section{Numerical Illustrations}
\label{sec:numerics}

In the following experiments, data was generated using a linear \scm{} \eqref{eq:sem:linear_sem} with a matrix $\semCoeffMat$ that is either fixed or random. For random \DAG{}-matrices, we follow \citet[section 4.1]{yu_daggnn_2019}: Let $\dNodes$ be the number of nodes in a \scm{}. Let $k$ be the expected number of edges in a randomly generated \DAG{}. Let $M$ be a random strictly subtriangular matrix where entries are drawn $\operatorname{Bernoulli}(2k/(\dNodes-1))$. Let $P$ be a random permutation matrix. Let $C$ be uniformly drawn from the interval $[0.5,2]$, and set $\semCoeffMat=P^\T (C\hadamard M) P$.

The random vector $\semNoise$ in \eqref{eq:sem:linear_sem} has elements with unit variance and are drawn independently as either Normal(0,1), Exp(1) or Gumbel(0,$6/\pi^2$)). Data was also centered before any other processing.

Throughout all runs, the nominal miscoverage level was set to $\confidenceLevel=5\%$ and $\dagTolerance=10^{-7}$.

\begin{remark}
     In the supplementary material, we study deviations from the linear data model, in which case the average causal effect \eqref{eq:def:averageCausalEffectTarget} of the optimal linear model is still defined.
\end{remark}
\begin{remark}
     In all cases when the data generator is a linear \scm{} with Gaussian noise, we apply Isserlis' theorem to equation \eqref{eq:mest:score_variance}, $\En\left[ \semVector_i \semVector_q \semVector_o \semVector_k \right]-
          \En\left[ \semVector_i \semVector_q \right]\En\left[ \semVector_o \semVector_k \right] =
          \En\left[ \semVector_i \semVector_o \right]\En\left[ \semVector_q \semVector_k \right]
          +\En\left[ \semVector_i \semVector_k \right]\En\left[ \semVector_q \semVector_o \right]
     $. This reduction is especially helpful in high dimensions, when $\dNodes$ is large.
\end{remark}

\subsection{Numerical Search Method} \label{subsection:numerical_search}
In the examples below, we construct the confidence interval \eqref{eq:confidence_set_for_ace} by numerically solving problem \eqref{eq:def:thetan}. Here we use the augmented Lagrangian method \citep{nocedal_numerical_2006}, but other search methods are possible as well.

We define the augmented Lagrangian and the equality converted constraint as
\begin{align}
     \label{eq:def:augLag}
     \augLag(\mEstParameter,\augLagSlack,\augLagLagMul, \augLagPen) = \En \left[ \mEstLoss_\mEstParameter(\semVector)\right] + \augLagLagMul \augLagContraint(\mEstParameter, \augLagSlack) + \frac{\augLagPen}{2}\augLagContraint(\mEstParameter, \augLagSlack)^2
\end{align}
\[\augLagContraint(\mEstParameter, \augLagSlack) = \hFun(\matop(L\mEstParameter))+\augLagSlack^2-\dagTolerance \]

The method alternates between the minimization over primal variables ($\mEstParameter$,$\augLagSlack$) and maximization over dual variables ($\augLagLagMul$), starting from a few initialization points, as explicated in  Algorithm~\ref{algo:augLag}.

\begin{algorithm}[ht!]
     %\SetAlgoLined
     \DontPrintSemicolon
     \KwIn{$\mEstParameter^0$,$\augLagSlack^0$,$\augLagPen^0$, $\augLagLagMul^0$, $\augLagMinImprovement$, $\augLagPenMul$,$\augLag$,$\augLagConstraintTol$,$\augLagPenMax$,$\augLagContraint$}
     \KwOut{$\mEstParameterEstN$ }
     $k=0$\;
     \nl\While{
          $\augLagContraint(\mEstParameter^{\augLagIter}, \augLagSlack^{\augLagIter}) > \augLagConstraintTol$
          \textbf{ and }
          $\augLagPen < \augLagPenMax$
          \label{algo:augLag:stopCondition}
     }{
          \nl$\mEstParameter^{k+1},\augLagSlack^{k+1} = \argmin_{\mEstParameter,\augLagSlack} \augLag(\mEstParameter,\augLagSlack,\augLagLagMul^\augLagIter, \augLagPen^\augLagIter)$\; \label{algo:augLag:inner_problem}
          $\augLagLagMul^{\augLagIter+1} = \augLagLagMul^{\augLagIter} + \augLagPen^\augLagIter \augLagContraint(\mEstParameter^{\augLagIter+1}, \augLagSlack^{\augLagIter+1})$ \;
          \eIf{
               $\augLagContraint(\mEstParameter^{\augLagIter+1}, \augLagSlack^{\augLagIter+1}) >  \augLagMinImprovement \augLagContraint(\mEstParameter^{\augLagIter}, \augLagSlack^{\augLagIter})$
          }{
               $\augLagPen^{\augLagIter+1} =\augLagPenMul \augLagPen^{\augLagIter}$ \;
          }{
               $\augLagPen^{\augLagIter+1} = \augLagPen^{\augLagIter}$\;
          }
          $k=k+1$\;
     }
     \KwRet{$\mEstParameterEstN=\mEstParameter^{\augLagIter+1}$}\;
     \caption{Augmented Lagrangian Method}
     \label{algo:augLag}
\end{algorithm}

The minimization problem on line~\ref{algo:augLag:inner_problem} is solved via the L-BFGS-B-implementation in the python library \texttt{scipy.optimize}, which in turn utilizes the 3.0 version of the FORTRAN library of \citet{zhu_algorithm_1997}. Since this is a local minimizer, we use the previous optimal primal variables $\mEstParameter^{k},\augLagSlack^{k}$ as the starting point.

The parameters have default values set to $\mEstParameter^0=0$, $\augLagSlack^0=10$, $\augLagPen^0=1$, $\augLagLagMul^0=0$, $\augLagMinImprovement=1/4$, $\augLagPenMul=2$, $\augLagConstraintTol=10^{-12}$, $\augLagPenMax = 10^{20}$. Note that $\augLagConstraintTol$ must be significantly smaller than $\dagTolerance$, which in turn should be smaller than $\dagToleranceMax$. Thus it is advisable to verify that the choice of $\augLagConstraintTol$ is sufficiently small in a given problem. The threshold $\augLagPenMax$ is introduced for numerical stability.

The augmented Lagrangian method is guaranteed to find a local minimizer $\mEstParameterEstN$, under a certain set of assumptions \citep[Theorem 17.6]{nocedal_numerical_2006}. One of these is constraint qualification at the minimizer, in this case demanding $\nabla \augLagContraint(\mEstParameter_*,\augLagSlack_*) \neq 0$ at the optimal primal variables $\mEstParameter_*,\augLagSlack_*$. For $\dagTolerance=0$ this do not hold, but it does so for $\dagTolerance > 0$, see Lemma~\ref{lemma:nonconvexWset} in the supplementary material for a proof. Finding the minimum for $\dagTolerance \to 0$ will thus require $\augLagPen \to \infty$, and we have introduced the stop condition $\augLagPenMax$ on line~\ref{algo:augLag:stopCondition} for practical reasons.

To compute $\averageCausalEffectTarget$ we replace $\En[..]$ in \eqref{eq:def:augLag} with $\E[..]$, which has a closed-form expression.


\subsection{Baseline Comparison}
\label{subsection:correctly_identify_adjustment}
\label{subsection:same_as_ols}

We first compare the proposed confidence interval $\averageCausalEffectSet_{\nData,\confidenceLevel}$ in \eqref{eq:confidence_set_for_ace} with a standard \OLS-based confidence interval $\regCoefficientSet_{\nData,\confidenceLevel}$ for \eqref{eq:partial_regression_coeff} that is computed using HC0 standard errors \citep{wooldridge_econometric_2010}. To use \OLS{} we must specify a set of control variables, which we take to be $\adjustmentVar$. When this set is valid, we expect $\averageCausalEffectSet_{\nData,\confidenceLevel}$ and $\regCoefficientSet_{\nData,\confidenceLevel}$ to be similar. When the set is invalid, we expect them to diverge.

We use the linear Gaussian data model with the matrix in \eqref{eq:sem:linear_sem} set to be either
\[\semCoeffMat '=\begin{bmatrix}
          0 & 0 & 1 & 0 \\
          0 & 0 & 1 & 0 \\
          0 & 0 & 0 & 0 \\
          1 & 1 & 0 & 0 \\
     \end{bmatrix} \text{ or } \semCoeffMat '' = \begin{bmatrix} 0& 0.4& 0\\ 0& 0& 0 \\ 0.7 & 0.2& 0 \end{bmatrix} \]

The graph of $\semCoeffMat '$ is illustrated in Figure~\ref{fig:collider_dag}, while Figure~\ref{fig:collider_asymptotics} demonstrates the ability of $\averageCausalEffectSet_{\nData,\confidenceLevel}$ to correctly infer $\averageCausalEffectTarget$ without specifying a set of control variables. By contrast, $\regCoefficientSet_{\nData,\confidenceLevel}$ is clearly biased from incorrectly controlling for the collider $\adjustmentVar_1$.

Corresponding results for $\semCoeffMat ''$ are shown in Figure~\ref{fig:fork_asymptotics}. As expected, the resulting intervals $\averageCausalEffectSet_{\nData,\confidenceLevel}$ and $\regCoefficientSet_{\nData,\confidenceLevel}$ are virtually identical since $\adjustmentVar$ constitutes a valid set of control variables.
\begin{figure}[htbp]
     \centering
     \begin{tikzpicture}[baseline]
    \pgfplotstableread[col sep=comma]{./data/3node_fork_summary.csv}{\datatable};
    \begin{semilogxaxis}[
        width=\columnwidth,
        height=.6\columnwidth,
        xlabel={No. of data points, $\nData$},
        ylabel={Causal parameter, $\averageCausalEffect$},
        xmin=90,
        xmax=11000,
        legend style={font=\scriptsize, at={(0.5,1.10)},
		anchor=south,legend columns=-1},
    ]
        \addplot+ [ only marks, mark=*, mark size=1pt, error bars/.cd, y dir=both, y explicit] table [x=m_obs, y=ace_value, y error=q_ace_standard_error] {\datatable};
        \addplot+ [only marks, mark=*, mark size= 1 pt, error bars/.cd, y dir=both, y explicit] table [x=m_obs, y=ols_value, y error=q_ols_standard_error] {\datatable};
        \addplot [no markers] table [x=m_obs, y=ace_circ] {\datatable};
        \legend{{$\averageCausalEffectSet_{\confidenceLevel,\nData}$},{$\regCoefficientSet_{\confidenceLevel,\nData}$},{$\averageCausalEffectTarget$}}
    \end{semilogxaxis}
\end{tikzpicture}

     \caption{$(1-\confidenceLevel)$-confidence intervals for $\averageCausalEffectTarget$ computed under a linear Gaussian \scm{} with matrix $\semCoeffMat ''$, for which $z$ is valid control variable.}
     \label{fig:fork_asymptotics}
\end{figure}



\subsection{Calibration and Normality}\label{subsection:calibration}
To assess the calibration of $\averageCausalEffectSet_{\confidenceLevel,\nData}$, we set $\nData$ to be $10^2$ or $10^4$ and generate repeated datasets from a linear Gaussian data model with matrix \[\semCoeffMat=\begin{bmatrix}
          0 & -2  & 1.6 & 0    \\
          0 & 0   & 0   & 0    \\
          0 & 1.2 & 0   & -0.5 \\
          0 & 0   & 0   & 0    \\
     \end{bmatrix}\]
corresponding to a graph illustrated in Figure~\ref{fig:calibration_dag}.
\begin{figure}
     \centering
     \begin{tikzpicture}[->,>=stealth',shorten >=1pt,auto,node distance=1.5cm,semithick]
  %\tikzstyle{every node}=[fill=none,draw=none]
 % https://tex.stackexchange.com/questions/445946/how-set-tikz-circle-radius-in-nodecircle
  \node[circle,draw, minimum size=20pt] (Z2) {$\adjustmentVar_1$};
  \node[circle,draw, minimum size=20pt] (X) [above right of=Z2] {$\decisionVar$};
  \node[circle,draw, minimum size=20pt] (Y) [below right of=Z2] {$\outcomeVar$};
  \node[circle,draw, minimum size=20pt] (Z1) [below right of=X] {$\adjustmentVar_2$};

  \path (Z1) edge   (Y)
  (X) edge   (Z1)
  (X) edge   (Y)
  (Y) edge   (Z2)
            ;
\end{tikzpicture}

     \caption{Causal structure of $\semCoeffMat$ in experiment for Calibration and Normality check, where $\adjustmentVar=[\adjustmentVar_1,\,\adjustmentVar_2]$ is  not a valid set of control variables.}
     \label{fig:calibration_dag}
\end{figure}

The coverage probability $\Prob ( \averageCausalEffectTarget \in  \averageCausalEffectSet_{\alpha, \nData}  )$ was estimated to be $94.6 \%$ and $94.9\%$ for $\nData=10^2$ and $10^4$, respectively, using $1000$ Monte Carlo simulations. This is close to $1-\alpha=95\%$ and corroborates Theorem~\ref{thm:confidence_set_for_ace}.
Figure~\ref{fig:calibration_qq} supports the result further by showing a Normality plot for the point estimate $\averageCausalEffectEstN$ over all Monte Carlo simulations.

\begin{figure}[htbp]
     \centering
     \begin{tikzpicture}[baseline]
    \pgfplotstableread[col sep=comma]{./data/calibration_qq.csv}{\datatable};
    \begin{axis}[
        width=\columnwidth,
        height=.6\columnwidth,
        xlabel={Theoretical Quantile},
        ylabel={Data Quantile},
        legend pos=south east,
        legend style={font=\scriptsize},
    ]
        \addplot [no marks, domain=-4:4] {x};
        \addplot+ [only marks] table [x=ace_n_theoretical_z, y=0] {\datatable};
        \addplot+ [only marks] table [x=ace_n_theoretical_z, y=1] {\datatable};
        \legend{,{$\nData=10^2$},{$\nData=10^4$}}
    \end{axis}
\end{tikzpicture}

     \caption{Normal probability plot for realizations of $\averageCausalEffectEstN$. Approximate normality is achieved even under moderate sample sizes.}
     \label{fig:calibration_qq}
\end{figure}

\subsection{Comparison With a Causal Discovery Method}

We compare our method with an alternative method of inferring the average causal effect by learning a linear \scm{} adjacency matrix $\semCoeffMat$ using DirectLiNGAM  \citep{shimizu_directlingam_2011, hyvarinen_pairwise_2013}. Then we can compute bootstrap confidence intervals, although they lack theoretical coverage guarantees. We used the official python implementation, version 1.5.1 from PyPI \url{https://pypi.org/project/lingam/1.5.1/}.

We generate a random adjacency matrix $\semCoeffMat$ for a graph on $\dNodes=10$ nodes and $k=1$, but with the random seed set to the lowest nonnegative integer that yielded a nonzero $\averageCausalEffect$ to make the comparison interesting. We use $\nData=10^{4}$ observations.

For LiNGAM, we computed the confidence interval (CI) using 100 bootstrap samples. For a comparable evaluation of its coverage, we considered the target quantity $\averageCausalEffectTarget$ to be the effect obtained when using LiNGAM with a large numbere of data points ($\nData'=10^6$). 100 Monte Carlo runs were used and the results are presented in Table~\ref{tab:lingam_compare}.

\begin{table}
     \centering
     \caption{Comparison of empirical coverage rate (CR) and the average width of the Confidence Interval (CI) for LiNGAM Bootstrap CI and the CI $\averageCausalEffectSet_{\confidenceLevel,\nData}$ proposed in this article. The nominal CR was set to exceed $1-\alpha = 95\%$}\label{tab:lingam_compare}
     \begin{tabular}{lllrr}
       \toprule
       Noise  & Method & CR    & Avg CI width & Avg $\averageCausalEffectEstN$ \\
       \midrule
       Normal & LiNGAM & 100\% & 2.01         & 0.64                           \\
              & our    & 99\%  & 0.15         & 1.79                           \\
       Exp    & LiNGAM & 92\%  & 0.08         & 1.77                           \\
              & our    & 100\% & 0.54         & 1.79                           \\
       Gumbel & LiNGAM & 85\%  & 0.07         & 1.77                           \\
              & our    & 100\% & 0.46         & 1.79                           \\
       \bottomrule
\end{tabular}

\end{table}

The results show that when data is Gaussian, our proposed method yields both well-calibrated and tighter CIs, than LiNGAM method which has a very wide CI. This expected as LiNGAM was designed for non-Gaussian data. Indeed, for the non-Gaussian examples, LiNGAM produces tighter CIs but they all undercover. By constrast, our method produces more conservative CIs that do not undercover and yield consistent inferences.






\subsection{Sensitivity with Respect to \DAG{} tolerance}
\label{subsection:numerics:sensitivity}
Let $\averageCausalEffectTarget(\dagTolerance)$ denote the average causal effect  \eqref{eq:def:averageCausalEffectTarget} when setting a specific value $\dagTolerance$ in \eqref{eq:def:semcoefftrue}. When data-generating process is given by a linear \scm  \eqref{eq:sem:linear_sem}, we have that the approximation gap $|\averageCausalEffect - \averageCausalEffectTarget(0)| = 0$, where $\averageCausalEffect$ is given by \eqref{eq:def:averageCausalEffectInSem}.
The gap should decrease with $\dagTolerance$ such that ideally $\lim_{\dagTolerance \to 0} |\averageCausalEffect - \averageCausalEffectTarget(\dagTolerance)|=0$  and, moreover. An analytical study is, however, beyond the scope of the tools considered herein and we therefore resort to a numerical sensitivity study.

First, we generate random \DAG{}-matrices $\semCoeffMat$. For every $\semCoeffMat$, we form the numerically approximation $\hat{\averageCausalEffectTarget}(\dagTolerance)$ by replacing $\En$ with the closed for expression for $\E$ in \eqref{eq:def:augLag}. In Figure~\ref{fig:epsilon-limit}, we illustrate the approximation gap $|\averageCausalEffect - \hat{\averageCausalEffectTarget}(\dagTolerance)|$. As expected the gap decreases sharply with $\dagTolerance$, until we reach finite precision effects arising mainly from the L-BFGS-B implementation.

\begin{figure}
     \centering
     \begin{tikzpicture}[baseline]
    \pgfplotstableread[col sep=comma]{./data/dagtol_pgfplots.csv}{\datatable};
    \begin{axis}[
        width=\columnwidth,
        height=.6\columnwidth,
        xlabel={$\dagTolerance$},
        ylabel={$|{\averageCausalEffect - \hat{\averageCausalEffectTarget}(\dagTolerance)}|$},
        xmode=log,  
        ymode=log,
        legend style={font=\scriptsize},
        xmin=1e-9,
        xmax=1e2,
    ]
    \foreach \y in {
1.1392723396082403
,0.1937917407495311
,1.285846198734098
,1.1420203278694752
,0.2970456915441977
,1.5075613943267605
,1.5455653003752525
,0.9413888160321484
,0.6245695864066603
,1.750413866544636
        }
        \addplot +[mark=none,gray,dashed,thick] coordinates {(\y, 1e-8) (\y, 1e0)};
    \foreach \k in {0,1,...,9}
        \addplot [mark=none,color=black] table [x=dag_tolerance, y=ace_abs_err-\k] {\datatable};
    \end{axis}
\end{tikzpicture}

     \caption{The error between $\averageCausalEffect$ \eqref{eq:def:averageCausalEffectInSem} for a randomly generated matrix $\semCoeffMat$ and the numerically evaluated $\hat{\averageCausalEffectTarget}(\dagTolerance)$ from \eqref{eq:def:averageCausalEffectTarget} and \eqref{eq:def:semcoefftrue}, over a range of $\dagTolerance$. Each solid line corresponds to the error for a randomly drawn matrix, with a corresponding value of $\dagToleranceMax$ shown as a vertical grey dashed line.
          For $\dagTolerance \lesssim 10^{-7}$ the numerical precision of our numerical solver limits the precision of the results.
     }
     \label{fig:epsilon-limit}
\end{figure}

For some of the random matrices, we notice that when $\dagTolerance > \dagToleranceMax$ we obtain unreliable approximations. A more detailed discussion is provided in Section~\ref{subsection:extra_sensitivity} in the supplementary material.

In the work of \citet{ng_ontheconvergence_2020}, it is shown that the convergence guarantees for augmented Lagrangian method do not hold and that its precision is finite as it terminates when the quadratic penalty $\rho$ approaches infinity --- in agreement both with our theoretical and experimental results.
